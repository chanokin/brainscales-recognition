\documentclass[11pt,a4paper]{article}
\usepackage[utf8]{inputenc}
\usepackage{amsmath}
\usepackage{amsfonts}
\usepackage{amssymb}
\usepackage{graphicx}
\usepackage{todonotes}

\graphicspath{{./images}}

\author{Garibaldi Pineda Garc{\'i}a}
\title{Report 19-11-2018}
\begin{document}
  \maketitle
  I arrived to the group on the 15th of November 2018; I've been settling in. 
  I almost have a fully working desktop computer!
  
  The objective is to make use of the BrainScaleS neuromorphic platform to do object recognition. 
  For this we will pursue odour identification taking inspiration from insect neural anatomy. 
  
  \section{Background reading}
  I read the Wikipedia entry on insect olfaction as a layman's introduction to the subject. 
  It mainly explains the structure of the olfactory pathway of an insect's nervous system. 
  
  I've started reading papers suggested by Mario Pannunzi:
    \subsection{Odor plumes and how insects use them}
    The authors describe the complexity of odour plumes and present a review of models. 
    There are 3 different aspects to the full plume structure model:
    \begin{itemize}
      \item Large scale. 
      This deals with the shape and odour strength (average). 
      Many models have been suggested for the direction aspect of the shape ranging from linear to wave-like. 
      Average strength is usually modelled as having an exponential decay with respect to distance.
      Insects use this aspect to orient themselves while finding the source. 
      
      \item Small scale. Odour concentration fluctuation; affects insect instantaneous response.
      Since the medium is turbulent, concentration changes along the plume. 
      Furthermore, insects' behaviour is modified depending on how dense the odour is (e.g. attraction vs. repulsion). 
      
      \item Time-average. How likely is an insect to smell the source.
      
    \end{itemize}
    In summary, odour perception is a complex problem as the signal is propagated through a medium (wind) which is usually turbulent. 
    The signal is thus unreliable and the medium does not commonly indicate the source's direction.
    Nonetheless, insects are able to locate stimuli emitters and react.
    \todo[inline]{Are we skipping this part in the simulation? We just assume antennas received a concentration X of odour Y?}
   \subsection{Learning pattern recognition and decision making in the insect brain}
   
\end{document}

