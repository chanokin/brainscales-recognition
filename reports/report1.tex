\documentclass[11pt,a4paper]{article}
\usepackage[utf8]{inputenc}
\usepackage{amsmath}
\usepackage{amsfonts}
\usepackage{amssymb}
\usepackage{graphicx}
\usepackage{todonotes}
\usepackage{url}
\graphicspath{{images/}}

\author{Garibaldi Pineda Garc{\'i}a}
\title{Report 26-11-2018}

\begin{document}

\maketitle

\section{Software environment}
\subsection{NEST and PyNN}
The default configuration for BrainScaleS (BSS) requires PyNN 0.7 (deprecated); if we want to do tests on the local host with Nest, we need to use version 2.2.2.\\


\url{https://github.com/nest/nest-releases/blob/master/nest-2.2.2.tar.gz}\\

\url{https://github.com/NeuralEnsemble/PyNN/releases/tag/0.7.5}\\

\subsection{Cypress}
Cypress is a spiking neural network simulation framework developed (and for use) with C++.
This allows for fast(er?) network building times and all the beauty that comes with a typed, compiled language.
This framework can serve as a PyNN wrapper allowing the use of  multiple back-end simulators (Nest, Brian, SpiNNaker, BrainScales, etc.); furthermore, the mindset for assembling networks is similar to PyNN 0.8 and up.
Unfortunately, Cypress supports Nest only with PyNN 0.8 which makes it a bit incompatible with the default configuration for BrainScales\\


\url{https://github.com/hbp-unibi/cypress}

\subsection{DEAP}

Distributed Evolutionary Algorithms in Python (DEAP) is a library which allows the user to easily perform multiple types of evolutionary algorithms. 
Since it is a Python library it would allow to easily integrate the PyNN-BSS workflow.\\ 

\url{https://deap.readthedocs.io/en/master/index.html}

\subsection{Open Beagle}
Open Beagle is an evolutionary computing framework for the C++ language. 
The design/usage philosophy is similar to DEAP and the interaction with Cypress would \underline{hopefully} be easy.\\

\url{https://github.com/chgagne/beagle}

%\subsection{ECF - Evolutionary Computation Framework}
%\url{http://ecf.zemris.fer.hr/}

\subsection{PaGMO}
PaGMO is a C++ (and Python) library for parallel optimization. 
The main advantage could be that we may use the Python interface for PyNN while keeping the high-performance C++ back-end.
Additionally, if this is not possible we could still use C++ with Cypress?
Another benefit is that we could jump from Genetic Algorithms to another optimization algorithm such as Particle Swarms or Simulated Annealing.\\

\url{https://esa.github.io/pagmo2/}

\end{document}
